% Options for packages loaded elsewhere
\PassOptionsToPackage{unicode}{hyperref}
\PassOptionsToPackage{hyphens}{url}
%
\documentclass[
]{article}
\usepackage[fontsize=7pt]{fontsize}

\usepackage[a4paper, margin=0.5in]{geometry}
\usepackage{lmodern}
\usepackage{amssymb,amsmath}
\usepackage{ifxetex,ifluatex}
\ifnum 0\ifxetex 1\fi\ifluatex 1\fi=0 % if pdftex
\usepackage[T1]{fontenc}
\usepackage[utf8]{inputenc}
\usepackage{textcomp} % provide euro and other symbols
\else % if luatex or xetex
\usepackage{unicode-math}
\defaultfontfeatures{Scale=MatchLowercase}
\defaultfontfeatures[\rmfamily]{Ligatures=TeX,Scale=1}
\fi
% Use upquote if available, for straight quotes in verbatim environments
\IfFileExists{upquote.sty}{\usepackage{upquote}}{}
\IfFileExists{microtype.sty}{% use microtype if available
  \usepackage[]{microtype}
  \UseMicrotypeSet[protrusion]{basicmath} % disable protrusion for tt fonts
}{}
\makeatletter
\@ifundefined{KOMAClassName}{% if non-KOMA class
  \IfFileExists{parskip.sty}{%
    \usepackage{parskip}
  }{% else
    \setlength{\parindent}{0pt}
    \setlength{\parskip}{6pt plus 2pt minus 1pt}}
}{% if KOMA class
  \KOMAoptions{parskip=half}}
\makeatother
\usepackage{xcolor}
\IfFileExists{xurl.sty}{\usepackage{xurl}}{} % add URL line breaks if available
\IfFileExists{bookmark.sty}{\usepackage{bookmark}}{\usepackage{hyperref}}
\hypersetup{
  hidelinks,
  pdfcreator={LaTeX via pandoc}}
\urlstyle{same} % disable monospaced font for URLs
\usepackage{longtable,booktabs}
% Correct order of tables after \paragraph or \subparagraph
\usepackage{etoolbox}
\makeatletter
\patchcmd\longtable{\par}{\if@noskipsec\mbox{}\fi\par}{}{}
\makeatother
% Allow footnotes in longtable head/foot
\IfFileExists{footnotehyper.sty}{\usepackage{footnotehyper}}{\usepackage{footnote}}
\makesavenoteenv{longtable}
\setlength{\emergencystretch}{3em} % prevent overfull lines
\providecommand{\tightlist}{%
  \setlength{\itemsep}{0pt}\setlength{\parskip}{0pt}}
\setcounter{secnumdepth}{-\maxdimen} % remove section numbering

% Define a variable
\newcommand{\columnA}{0.06}
\newcommand{\columnB}{0.18}
\newcommand{\columnC}{0.33}
\newcommand{\columnD}{0.33}

\author{}
\date{}

\begin{document}

\renewcommand*{\arraystretch}{1.6}
\begin{longtable}[]{@{}llll@{}}
\toprule
\begin{minipage}[b]{\columnA\columnwidth}\raggedright
\textbf{Symbol}\strut
\end{minipage} & \begin{minipage}[b]{\columnB\columnwidth}\raggedright
\textbf{Name}\strut
\end{minipage} & \begin{minipage}[b]{\columnC\columnwidth}\raggedright
\textbf{Formula}\strut
\end{minipage} & \begin{minipage}[b]{\columnD\columnwidth}\raggedright
\textbf{Description / Example}\strut
\end{minipage}\tabularnewline
\midrule
\endhead
\begin{minipage}[t]{\columnA\columnwidth}\raggedright
Homography\strut
\end{minipage} & \begin{minipage}[t]{\columnB\columnwidth}\raggedright
\strut
\end{minipage} & \begin{minipage}[t]{\columnC\columnwidth}\raggedright
\strut
\end{minipage} & \begin{minipage}[t]{\columnD\columnwidth}\raggedright
\strut
\end{minipage}\hline\tabularnewline
\begin{minipage}[t]{\columnA\columnwidth}\raggedright
\(\tilde{X_i^c}\)\strut
\end{minipage} & \begin{minipage}[t]{\columnB\columnwidth}\raggedright
Image coordinates\strut
\end{minipage} & \begin{minipage}[t]{\columnC\columnwidth}\raggedright
\(\tilde{X_i^c} \backsim\ \tilde{H_b^c} \cdotp \tilde{p}_i^b\)\strut
\end{minipage} & \begin{minipage}[t]{\columnD\columnwidth}\raggedright
\strut
\end{minipage}\hline\tabularnewline
\begin{minipage}[t]{\columnA\columnwidth}\raggedright
\(K_c\)\strut
\end{minipage} & \begin{minipage}[t]{\columnB\columnwidth}\raggedright
Calibration matrix, Camera matrix\strut
\end{minipage} & \begin{minipage}[t]{\columnC\columnwidth}\raggedright
\(K_c = \begin{bmatrix} f_x & 0 & c_x \\ 0 & f_y & c_y \\ 0 & 0 & 1 \end{bmatrix}\)\strut
\end{minipage} & \begin{minipage}[t]{\columnD\columnwidth}\raggedright
If \(c_x=c_y=0\) the camera produces centerd images\strut
\end{minipage}\hline\tabularnewline
\begin{minipage}[t]{\columnA\columnwidth}\raggedright
\(\tilde{x}\)\strut
\end{minipage} & \begin{minipage}[t]{\columnB\columnwidth}\raggedright
homogenous transformation\strut
\end{minipage} & \begin{minipage}[t]{\columnC\columnwidth}\raggedright
\(\tilde{x}= K_c*\begin{bmatrix} x\\y\\z \end{bmatrix}\)\strut
\end{minipage} & \begin{minipage}[t]{\columnD\columnwidth}\raggedright
for direction of light rays f.e.\strut
\end{minipage}\hline\tabularnewline
\begin{minipage}[t]{\columnA\columnwidth}\raggedright
\(\tilde{p}_i^b\)\strut
\end{minipage} & \begin{minipage}[t]{\columnB\columnwidth}\raggedright
Object point\strut
\end{minipage} & \begin{minipage}[t]{\columnC\columnwidth}\raggedright
\strut
\end{minipage} & \begin{minipage}[t]{\columnD\columnwidth}\raggedright
Coordinates on the planar object\strut
\end{minipage}\hline\tabularnewline
\begin{minipage}[t]{\columnA\columnwidth}\raggedright
\(E_b^c\)\strut
\end{minipage} & \begin{minipage}[t]{\columnB\columnwidth}\raggedright
Camera extrinsic matrix\strut
\end{minipage} & \begin{minipage}[t]{\columnC\columnwidth}\raggedright
\(E_b^c = \begin{bmatrix} R_b^c & t_{cb}^c \\ \begin{bmatrix} 0 & 0 & 0 \end{bmatrix}& 1 \end{bmatrix}\)\strut
\end{minipage} & \begin{minipage}[t]{\columnD\columnwidth}\raggedright
Relative pose between object and camera\strut
\end{minipage}\hline\tabularnewline
\begin{minipage}[t]{\columnA\columnwidth}\raggedright
\(E_c^g\)\strut
\end{minipage} & \begin{minipage}[t]{\columnB\columnwidth}\raggedright
Inverse rigid motion matrix\strut
\end{minipage} & \begin{minipage}[t]{\columnC\columnwidth}\raggedright
\(E_c^g=(E_g^c)^{-1}=\begin{bmatrix} R_g^c&t_{cg}^c\\\vec{0}&1 \end{bmatrix}=\)
\(\begin{bmatrix} R_c^g&-R_c^gt_{cg}^c\\\vec{0}&1 \end{bmatrix}\)\strut
\end{minipage} & \begin{minipage}[t]{\columnD\columnwidth}\raggedright
\strut
\end{minipage}\hline\tabularnewline
\begin{minipage}[t]{\columnA\columnwidth}\raggedright
\(R_b^c\)\strut
\end{minipage} & \begin{minipage}[t]{\columnB\columnwidth}\raggedright
Rotation matrix\strut
\end{minipage} & \begin{minipage}[t]{\columnC\columnwidth}\raggedright
\(R_b^c=\begin{bmatrix}r_x & r_y & r_z\end{bmatrix}\)\strut
\end{minipage} & \begin{minipage}[t]{\columnD\columnwidth}\raggedright
\strut
\end{minipage}\hline\tabularnewline
\begin{minipage}[t]{\columnA\columnwidth}\raggedright
\(t_{cb}^c\)\strut
\end{minipage} & \begin{minipage}[t]{\columnB\columnwidth}\raggedright
Translation vector\strut
\end{minipage} & \begin{minipage}[t]{\columnC\columnwidth}\raggedright
\(t_c^b=p^c-R_b^c*p^b\)\strut
\end{minipage} & \begin{minipage}[t]{\columnD\columnwidth}\raggedright
\strut
\end{minipage}\hline\tabularnewline
\begin{minipage}[t]{\columnA\columnwidth}\raggedright
\(\vec{E}_g\)\strut
\end{minipage} & \begin{minipage}[t]{\columnB\columnwidth}\raggedright
Coordinate frame basis\strut
\end{minipage} & \begin{minipage}[t]{\columnC\columnwidth}\raggedright
\(\vec{E}_g = \begin{bmatrix} \vec{e}_{g,x} & \vec{e}_{g,y} & \vec{e}_{g,z} \end{bmatrix}\)\strut
\end{minipage} & \begin{minipage}[t]{\columnD\columnwidth}\raggedright
A coordinate frame consists of a basis and and an origin\strut
\end{minipage}\hline\tabularnewline
\begin{minipage}[t]{\columnA\columnwidth}\raggedright
\(\vec{o}_g\)\strut
\end{minipage} & \begin{minipage}[t]{\columnB\columnwidth}\raggedright
Origin of the coordinate frame\strut
\end{minipage} & \begin{minipage}[t]{\columnC\columnwidth}\raggedright
\(\vec{o}\)\strut
\end{minipage} & \begin{minipage}[t]{\columnD\columnwidth}\raggedright
\strut
\end{minipage}\hline\tabularnewline
\begin{minipage}[t]{\columnA\columnwidth}\raggedright
\(\vec{p}\)\strut
\end{minipage} & \begin{minipage}[t]{\columnB\columnwidth}\raggedright
Point in the coordinate frame\strut
\end{minipage} & \begin{minipage}[t]{\columnC\columnwidth}\raggedright
\(\vec{p} = \vec{e}_{g,x} \cdotp p_x^g + \vec{e}_{g,y} \cdotp p_y^g + \vec{e}_{g,z} \cdotp p_z^g + \vec{o}_g\)\strut
\end{minipage} & \begin{minipage}[t]{\columnD\columnwidth}\raggedright
\strut
\end{minipage}\hline\tabularnewline
\begin{minipage}[t]{\columnA\columnwidth}\raggedright
\(\vec{p}\)\strut
\end{minipage} & \begin{minipage}[t]{\columnB\columnwidth}\raggedright
Point in the coordinate frame\strut
\end{minipage} & \begin{minipage}[t]{\columnC\columnwidth}\raggedright
\(\vec{p} = \vec{E}_g \cdotp p^g + \vec{o}_g\)\strut
\end{minipage} & \begin{minipage}[t]{\columnD\columnwidth}\raggedright
\strut
\end{minipage}\hline\tabularnewline
\begin{minipage}[t]{\columnA\columnwidth}\raggedright
\(H_b^c\)\strut
\end{minipage} & \begin{minipage}[t]{\columnB\columnwidth}\raggedright
Homography matrix\strut
\end{minipage} & \begin{minipage}[t]{\columnC\columnwidth}\raggedright
\(H_b^c \backsim\ K_c \cdotp \begin{bmatrix} r^c_{b,x} & r_{b,y}^c & t_{cb}^c \end{bmatrix}\)\strut
\end{minipage} & \begin{minipage}[t]{\columnD\columnwidth}\raggedright
\strut
\end{minipage}\hline\tabularnewline
\begin{minipage}[t]{\columnA\columnwidth}\raggedright
\(r_{b,x}^c\)\strut
\end{minipage} & \begin{minipage}[t]{\columnB\columnwidth}\raggedright
Rotation vector x\strut
\end{minipage} & \begin{minipage}[t]{\columnC\columnwidth}\raggedright
\(r_{b,x}^c = \begin{bmatrix} 0 & -t_{cb,z}^c & t_{cb,y}^c \end{bmatrix}\)\strut
\end{minipage} & \begin{minipage}[t]{\columnD\columnwidth}\raggedright
\strut
\end{minipage}\hline\tabularnewline
\begin{minipage}[t]{\columnA\columnwidth}\raggedright
\strut
\end{minipage} & \begin{minipage}[t]{\columnB\columnwidth}\raggedright
Object point to image point\strut
\end{minipage} & \begin{minipage}[t]{\columnC\columnwidth}\raggedright
\(\begin{bmatrix} x_{s,i} \\ y_{s,i} \\ 1 \end{bmatrix} = H_b^c \cdotp \tilde{p}_i^b\)\strut
\end{minipage} & \begin{minipage}[t]{\columnD\columnwidth}\raggedright
\strut
\end{minipage}\hline\tabularnewline
\begin{minipage}[t]{\columnA\columnwidth}\raggedright
Hough~transform\strut
\end{minipage} & \begin{minipage}[t]{\columnB\columnwidth}\raggedright
\strut
\end{minipage} & \begin{minipage}[t]{\columnC\columnwidth}\raggedright
\strut
\end{minipage} & \begin{minipage}[t]{\columnD\columnwidth}\raggedright
\strut
\end{minipage}\hline\tabularnewline
\begin{minipage}[t]{\columnA\columnwidth}\raggedright
\(x_i, y_i\)\strut
\end{minipage} & \begin{minipage}[t]{\columnB\columnwidth}\raggedright
Image space coordinates\strut
\end{minipage} & \begin{minipage}[t]{\columnC\columnwidth}\raggedright
\(y_i = m \cdotp x_i + c \Leftrightarrow c = - m \cdotp x_i + y_i\)\strut
\end{minipage} & \begin{minipage}[t]{\columnD\columnwidth}\raggedright
Converted to parameter space, lines\strut
\end{minipage}\hline\tabularnewline
\begin{minipage}[t]{\columnA\columnwidth}\raggedright
\(\theta\)\strut
\end{minipage} & \begin{minipage}[t]{\columnB\columnwidth}\raggedright
Angle of point\strut
\end{minipage} & \begin{minipage}[t]{\columnC\columnwidth}\raggedright
\(\theta\)\strut
\end{minipage} & \begin{minipage}[t]{\columnD\columnwidth}\raggedright
angle between \(x\) and line in parameter space\strut
\end{minipage}\hline\tabularnewline
\begin{minipage}[t]{\columnA\columnwidth}\raggedright
\(\rho\)\strut
\end{minipage} & \begin{minipage}[t]{\columnB\columnwidth}\raggedright
Proper Line Parametrization\strut
\end{minipage} & \begin{minipage}[t]{\columnC\columnwidth}\raggedright
\(\rho = x \cos(\theta) + y \sin(\theta)\)\strut
\end{minipage} & \begin{minipage}[t]{\columnD\columnwidth}\raggedright
length of line\strut
\end{minipage}\hline\tabularnewline
\begin{minipage}[t]{\columnA\columnwidth}\raggedright
\(\rho\)\strut
\end{minipage} & \begin{minipage}[t]{\columnB\columnwidth}\raggedright
\strut
\end{minipage} & \begin{minipage}[t]{\columnC\columnwidth}\raggedright
\(\rho =\begin{bmatrix}x \\ y\end{bmatrix}^t \cdot n\)\strut
\end{minipage} & \begin{minipage}[t]{\columnD\columnwidth}\raggedright
Test if a point is on a line\strut
\end{minipage}\hline\tabularnewline
\begin{minipage}[t]{\columnA\columnwidth}\raggedright
\(n\)\strut
\end{minipage} & \begin{minipage}[t]{\columnB\columnwidth}\raggedright
normal vector\strut
\end{minipage} & \begin{minipage}[t]{\columnC\columnwidth}\raggedright
\(n = \begin{bmatrix} \cos(\theta) \\ \sin(\theta)\end{bmatrix}\)\strut
\end{minipage} & \begin{minipage}[t]{\columnD\columnwidth}\raggedright
Defined by the \(\theta\) angle\strut
\end{minipage}\hline\tabularnewline
\begin{minipage}[t]{\columnA\columnwidth}\raggedright
RANSAC~probabilities\strut
\end{minipage} & \begin{minipage}[t]{\columnB\columnwidth}\raggedright
\strut
\end{minipage} & \begin{minipage}[t]{\columnC\columnwidth}\raggedright
\strut
\end{minipage} & \begin{minipage}[t]{\columnD\columnwidth}\raggedright
\strut
\end{minipage}\hline\tabularnewline
\begin{minipage}[t]{\columnA\columnwidth}\raggedright
\(\epsilon\)\strut
\end{minipage} & \begin{minipage}[t]{\columnB\columnwidth}\raggedright
Probability of picking an outlier\strut
\end{minipage} & \begin{minipage}[t]{\columnC\columnwidth}\raggedright
\(\epsilon = \dfrac{N_{outliers}}{N_{inliers} + N_{outliers}}\)\strut
\end{minipage} & \begin{minipage}[t]{\columnD\columnwidth}\raggedright
with \(N\) = no of, \(s\) = points, \(n\) = no. of trials\strut
\end{minipage}\hline\tabularnewline
\begin{minipage}[t]{\columnA\columnwidth}\raggedright
\strut
\end{minipage} & \begin{minipage}[t]{\columnB\columnwidth}\raggedright
probability of picking individual inlier\strut
\end{minipage} & \begin{minipage}[t]{\columnC\columnwidth}\raggedright
\(p=1-\epsilon\)\strut
\end{minipage} & \begin{minipage}[t]{\columnD\columnwidth}\raggedright
\strut
\end{minipage}\hline\tabularnewline
\begin{minipage}[t]{\columnA\columnwidth}\raggedright
\strut
\end{minipage} & \begin{minipage}[t]{\columnB\columnwidth}\raggedright
probability of picking \(s\) inliers in sequence\strut
\end{minipage} & \begin{minipage}[t]{\columnC\columnwidth}\raggedright
\(p=(1-\epsilon)^s\)\strut
\end{minipage} & \begin{minipage}[t]{\columnD\columnwidth}\raggedright
\strut
\end{minipage}\hline\tabularnewline
\begin{minipage}[t]{\columnA\columnwidth}\raggedright
\strut
\end{minipage} & \begin{minipage}[t]{\columnB\columnwidth}\raggedright
probability of not picking \(s\) inliers in sequence\strut
\end{minipage} & \begin{minipage}[t]{\columnC\columnwidth}\raggedright
\(p=1-(1-\epsilon)^s\)\strut
\end{minipage} & \begin{minipage}[t]{\columnD\columnwidth}\raggedright
\strut
\end{minipage}\hline\tabularnewline
\begin{minipage}[t]{\columnA\columnwidth}\raggedright
\strut
\end{minipage} & \begin{minipage}[t]{\columnB\columnwidth}\raggedright
probability of not picking \(s\) inliers in sequence of \(n\)
trials\strut
\end{minipage} & \begin{minipage}[t]{\columnC\columnwidth}\raggedright
\(p=(1-(1-\epsilon)^s)^n\)\strut
\end{minipage} & \begin{minipage}[t]{\columnD\columnwidth}\raggedright
\strut
\end{minipage}\hline\tabularnewline
\begin{minipage}[t]{\columnA\columnwidth}\raggedright
\strut
\end{minipage} & \begin{minipage}[t]{\columnB\columnwidth}\raggedright
probability of picking at least in one of \(n\) trials \(s\) inliers in
sequence\strut
\end{minipage} & \begin{minipage}[t]{\columnC\columnwidth}\raggedright
\(p_{success}=1-(1-(1-\epsilon)^s)^n\)\strut
\end{minipage} & \begin{minipage}[t]{\columnD\columnwidth}\raggedright
for lines 2 , for circles 3 points are needed\strut
\end{minipage}\hline\tabularnewline
\begin{minipage}[t]{\columnA\columnwidth}\raggedright
\strut
\end{minipage} & \begin{minipage}[t]{\columnB\columnwidth}\raggedright
expected number of trials needed\strut
\end{minipage} & \begin{minipage}[t]{\columnC\columnwidth}\raggedright
\(n=\dfrac{log(1-p_{success})}{log(1-(1-\epsilon)^s)}\)\strut
\end{minipage} & \begin{minipage}[t]{\columnD\columnwidth}\raggedright
\strut
\end{minipage}\hline\tabularnewline
\begin{minipage}[t]{\columnA\columnwidth}\raggedright
Geometric transformation\strut
\end{minipage} & \begin{minipage}[t]{\columnB\columnwidth}\raggedright
\strut
\end{minipage} & \begin{minipage}[t]{\columnC\columnwidth}\raggedright
\strut
\end{minipage} & \begin{minipage}[t]{\columnD\columnwidth}\raggedright
\strut
\end{minipage}\hline\tabularnewline
\begin{minipage}[t]{\columnA\columnwidth}\raggedright
\(\tilde{x}\)\strut
\end{minipage} & \begin{minipage}[t]{\columnB\columnwidth}\raggedright
Intersection of two lines\strut
\end{minipage} & \begin{minipage}[t]{\columnC\columnwidth}\raggedright
\(\tilde{x}=\tilde{I_1}\times \tilde{I_2}\)\strut
\end{minipage} & \begin{minipage}[t]{\columnD\columnwidth}\raggedright
cross product of two lines defines their intersection\strut
\end{minipage}\hline\tabularnewline
\begin{minipage}[t]{\columnA\columnwidth}\raggedright
\(\tilde{I}\)\strut
\end{minipage} & \begin{minipage}[t]{\columnB\columnwidth}\raggedright
two points lie on the line\strut
\end{minipage} & \begin{minipage}[t]{\columnC\columnwidth}\raggedright
\(\tilde{I}=\tilde{x_1}\times \tilde{x_2}\)\strut
\end{minipage} & \begin{minipage}[t]{\columnD\columnwidth}\raggedright
cross product of two points define their collective line\strut
\end{minipage}\hline\tabularnewline
\begin{minipage}[t]{\columnA\columnwidth}\raggedright
Matrix basics\strut
\end{minipage} & \begin{minipage}[t]{\columnB\columnwidth}\raggedright
\strut
\end{minipage} & \begin{minipage}[t]{\columnC\columnwidth}\raggedright
\strut
\end{minipage} & \begin{minipage}[t]{\columnD\columnwidth}\raggedright
\strut
\end{minipage}\hline\tabularnewline
\begin{minipage}[t]{\columnA\columnwidth}\raggedright
\(E\)\strut
\end{minipage} & \begin{minipage}[t]{\columnB\columnwidth}\raggedright
unit matrix\strut
\end{minipage} & \begin{minipage}[t]{\columnC\columnwidth}\raggedright
\(\begin{bmatrix} 1&0&0\\0&1&0\\0&0&1 \end{bmatrix}\)\strut
\end{minipage} & \begin{minipage}[t]{\columnD\columnwidth}\raggedright
\strut
\end{minipage}\hline\tabularnewline
\begin{minipage}[t]{\columnA\columnwidth}\raggedright
\(R^{-1}\)\strut
\end{minipage} & \begin{minipage}[t]{\columnB\columnwidth}\raggedright
Inverse rotational matrix\strut
\end{minipage} & \begin{minipage}[t]{\columnC\columnwidth}\raggedright
\(R^{-1} = R^T\)\strut
\end{minipage} & \begin{minipage}[t]{\columnD\columnwidth}\raggedright
\strut
\end{minipage}\hline\tabularnewline
\begin{minipage}[t]{\columnA\columnwidth}\raggedright
\(R_x\)\strut
\end{minipage} & \begin{minipage}[t]{\columnB\columnwidth}\raggedright
Rotational matrix around x\strut
\end{minipage} & \begin{minipage}[t]{\columnC\columnwidth}\raggedright
\(R_x=\begin{bmatrix} 1&0&0\\0&\cos&-\sin\\0&\sin&\cos \end{bmatrix}\)\strut
\end{minipage} & \begin{minipage}[t]{\columnD\columnwidth}\raggedright
\strut
\end{minipage}\hline\tabularnewline
\begin{minipage}[t]{\columnA\columnwidth}\raggedright
\(R_y\)\strut
\end{minipage} & \begin{minipage}[t]{\columnB\columnwidth}\raggedright
Rotational matrix around x\strut
\end{minipage} & \begin{minipage}[t]{\columnC\columnwidth}\raggedright
\(R_y=\begin{bmatrix} \cos&0&\sin\\0&1&0\\-\sin&0&\cos \end{bmatrix}\)\strut
\end{minipage} & \begin{minipage}[t]{\columnD\columnwidth}\raggedright
\strut
\end{minipage}\hline\tabularnewline
\begin{minipage}[t]{\columnA\columnwidth}\raggedright
\(R_z\)\strut
\end{minipage} & \begin{minipage}[t]{\columnB\columnwidth}\raggedright
Rotational matrix around x\strut
\end{minipage} & \begin{minipage}[t]{\columnC\columnwidth}\raggedright
\(R_z=\begin{bmatrix} \cos&-\sin&0\\\sin&\cos&0\\0&0&1 \end{bmatrix}\)\strut
\end{minipage} & \begin{minipage}[t]{\columnD\columnwidth}\raggedright
\strut
\end{minipage}\hline\tabularnewline
\begin{minipage}[t]{\columnA\columnwidth}\raggedright
Camera calculations\strut
\end{minipage} & \begin{minipage}[t]{\columnB\columnwidth}\raggedright
\strut
\end{minipage} & \begin{minipage}[t]{\columnC\columnwidth}\raggedright
\strut
\end{minipage} & \begin{minipage}[t]{\columnD\columnwidth}\raggedright
\strut
\end{minipage}\hline\tabularnewline
\begin{minipage}[t]{\columnA\columnwidth}\raggedright
\(F\)\strut
\end{minipage} & \begin{minipage}[t]{\columnB\columnwidth}\raggedright
Focal length in {[}\(mm\){]}\strut
\end{minipage} & \begin{minipage}[t]{\columnC\columnwidth}\raggedright
\(F=\dfrac{L*x_{chip}}{2*W}\)\strut
\end{minipage} & \begin{minipage}[t]{\columnD\columnwidth}\raggedright
\(L=\) Length from sensor to object, \(W=\) Width from sensor to object,
all in {[}\(mm\){]}\strut
\end{minipage}\hline\tabularnewline
\begin{minipage}[t]{\columnA\columnwidth}\raggedright
\(f_x\)\strut
\end{minipage} & \begin{minipage}[t]{\columnB\columnwidth}\raggedright
Focal length in {[}\(\dfrac{pixel}{mm}\){]}\strut
\end{minipage} & \begin{minipage}[t]{\columnC\columnwidth}\raggedright
\(f_x = \dfrac{F*x_{pixel}}{x_{chip}}\)\strut
\end{minipage} & \begin{minipage}[t]{\columnD\columnwidth}\raggedright
\(x_{pixel}=\) pixel in \(x\)-direction \(x_{chip}=\) length of sensor
chip in \(x\)-direction, analog for \(f_y\) with \(y_{pixel}\) and
\(y_{chip}\)\strut
\end{minipage}\hline\tabularnewline
\begin{minipage}[t]{\columnA\columnwidth}\raggedright
\(f_x, f_y\)\strut
\end{minipage} & \begin{minipage}[t]{\columnB\columnwidth}\raggedright
Focal length (Assumption)\strut
\end{minipage} & \begin{minipage}[t]{\columnC\columnwidth}\raggedright
\(f_x = f_y = f\)\strut
\end{minipage} & \begin{minipage}[t]{\columnD\columnwidth}\raggedright
In some calcuations just assume: focal length is the same in both
directions\strut
\end{minipage}\hline\tabularnewline
\begin{minipage}[t]{\columnA\columnwidth}\raggedright
\(c_x, c_y\)\strut
\end{minipage} & \begin{minipage}[t]{\columnB\columnwidth}\raggedright
image center coordinates\strut
\end{minipage} & \begin{minipage}[t]{\columnC\columnwidth}\raggedright
\(c_x=\dfrac{x_{pixel}}{2}\)\strut
\end{minipage} & \begin{minipage}[t]{\columnD\columnwidth}\raggedright
optical axis pointing perpendicularly through sensor chip center, analog
for \(f_y\) with \(y_{pixel}\) and \(y_{chip}\)\strut
\end{minipage}\hline\tabularnewline
\bottomrule
\end{longtable}

\end{document}
